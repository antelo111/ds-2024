\documentclass[UTF8]{ctexart}
\usepackage{geometry, CJKutf8}
\geometry{margin=1.5cm, vmargin={0pt,1cm}}
\setlength{\topmargin}{-1cm}
\setlength{\paperheight}{29.7cm}
\setlength{\textheight}{25.3cm}

% useful packages.
\usepackage{xcolor}
\usepackage{amsfonts}
\usepackage{amsmath}
\usepackage{amssymb}
\usepackage{amsthm}
\usepackage{enumerate}
\usepackage{graphicx}
\usepackage{multicol}
\usepackage{fancyhdr}
\usepackage{layout}
\usepackage{listings}
\usepackage{float, caption}

\lstset{  
  basicstyle=\ttfamily,  
  backgroundcolor=\color{white},  
  frame=single,  
  framesep=5pt,  
  rulecolor=\color{black},   
  breaklines=true,  
  xleftmargin=\parindent,  
  xrightmargin=\parindent  
}  

% some common command
\newcommand{\dif}{\mathrm{d}}
\newcommand{\avg}[1]{\left\langle #1 \right\rangle}
\newcommand{\difFrac}[2]{\frac{\dif #1}{\dif #2}}
\newcommand{\pdfFrac}[2]{\frac{\partial #1}{\partial #2}}
\newcommand{\OFL}{\mathrm{OFL}}
\newcommand{\UFL}{\mathrm{UFL}}
\newcommand{\fl}{\mathrm{fl}}
\newcommand{\op}{\odot}
\newcommand{\Eabs}{E_{\mathrm{abs}}}
\newcommand{\Erel}{E_{\mathrm{rel}}}

\begin{document}

\pagestyle{fancy}
\fancyhead{}
\lhead{骆弘毅, 3210103287}
\chead{数据结构与算法项目}
\rhead{Dec.28th, 2024}

\section{四则运算计算器设计思路}
\indent 该项目目标是实现一个可以支持括号的四则运算计算器,其本质是对中缀表达式进行求值。因此基础的思路有两种:其一,我们知道逆波兰表达式在计算机中易于实现求值,我们仅需从左向右依次将数值入栈,每当遇到运算符时,弹出两个栈顶元素并计算,最后重新压入栈中。最终栈中元素即为所求结果。因此可以将中缀表达式转化为逆波兰表达式也就是后缀表达式,便可以方便的进行求值。其二,考虑到转为后缀表达式再求值其实涉及到了一些不必要的字符串操作,我们可以将上述两个步骤做合并,也就是在转后缀表达式的过程中直接进行求值计算。具体来说,我们先建立两个栈,一个数值栈一个符号栈,我们从左向右扫描表达式,当遇到数值时将其压入数值栈中,而遇到符号时,若遇到括号界限符时,将左括号压入符号栈,而当遇到右括号时,不入栈而是将符号栈中元素弹出,每当弹出一个元素时,同时从数值栈中弹出两个对应的运算数,求值并重新压入数值栈,直到符号栈中的左括号被弹出。而遇到运算符时,我们不直接进行运算而是和栈顶元素比较运算优先级,当该运算符优先级低于或等于栈顶元素优先级时,弹出栈顶符号,同时弹出对应操作数并运算,最后重新入栈,直到符号栈空或是栈顶元素优先级低于当前符号。不断重复以上操作后,我们用和后缀表达式一样的方法对两个栈做最后一步处理,便得到了想要的答案。\\
\indent 同时该计算器实现了对非法输入的检测,具体来说有括号不匹配检测,除0检测,和符号非法使用检测等等,而所有检测的实现均使用了单独的函数,从而方便后续维护处理。括号不匹配检测的方法为在扫描过程中实时记录左括号和右括号的数量,当右括号数量大于左括号时则直接返回false;除0检测使用的是字符串匹配,检测字符串中是否有\/0这个模式串,若出现则进行进一步判断,若下一个字符为运算符或界限符则返回false,如果为\.则说明后续为介于0和1之间的小数,不进行报错;对于符号非法使用,只需检测首尾是否为运算符以及相邻字符是否均为运算符即可。

\section{测试结果}
\begin{center}
\begin{tabular}{|c|c|c|}
\hline
测试目的 & 表达式 & 输出结果 \\
\hline
括号错用 & )1+2+3( & ILLEGAL \\
\hline
括号不匹配 & (1*(2+3)+4 & ILLEGAL \\
\hline
符号使用不当 & +1+2 & ILLEGAL \\
\hline
符号使用不当 & 2+1+ & ILLEGAL \\
\hline
符号使用不当 & 35++67 & ILLEGAL \\
\hline
除0 & 1/0 & ILLEGAL \\
\hline
带有'/0' 的正常输入 & 28/0.28 & 100 \\
\hline
正常输入测试 & (1+2+3)/2 & 3 \\
\hline
较为复杂的输入测试 & (1+3*5)/(4-4/2) & 8 \\
\hline
实数运算测试 & (0.2+0.9)/(12-1)+1 & 1.1 \\
\hline
\end{tabular}
\end{center}

\end{document}

%%% Local Variables: 
%%% mode: latex
%%% TeX-master: t
%%% End: 
