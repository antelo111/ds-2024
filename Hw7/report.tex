\documentclass[UTF8]{ctexart}
\usepackage{geometry, CJKutf8}
\geometry{margin=1.5cm, vmargin={0pt,1cm}}
\setlength{\topmargin}{-1cm}
\setlength{\paperheight}{29.7cm}
\setlength{\textheight}{25.3cm}

% useful packages.
\usepackage{xcolor}
\usepackage{amsfonts}
\usepackage{amsmath}
\usepackage{amssymb}
\usepackage{amsthm}
\usepackage{enumerate}
\usepackage{graphicx}
\usepackage{multicol}
\usepackage{fancyhdr}
\usepackage{layout}
\usepackage{listings}
\usepackage{float, caption}

\lstset{  
  basicstyle=\ttfamily,  
  backgroundcolor=\color{white},  
  frame=single,  
  framesep=5pt,  
  rulecolor=\color{black},   
  breaklines=true,  
  xleftmargin=\parindent,  
  xrightmargin=\parindent  
}  

% some common command
\newcommand{\dif}{\mathrm{d}}
\newcommand{\avg}[1]{\left\langle #1 \right\rangle}
\newcommand{\difFrac}[2]{\frac{\dif #1}{\dif #2}}
\newcommand{\pdfFrac}[2]{\frac{\partial #1}{\partial #2}}
\newcommand{\OFL}{\mathrm{OFL}}
\newcommand{\UFL}{\mathrm{UFL}}
\newcommand{\fl}{\mathrm{fl}}
\newcommand{\op}{\odot}
\newcommand{\Eabs}{E_{\mathrm{abs}}}
\newcommand{\Erel}{E_{\mathrm{rel}}}

\begin{document}

\pagestyle{fancy}
\fancyhead{}
\lhead{骆弘毅, 3210103287}
\chead{数据结构与算法第七次作业}
\rhead{Nov.30th, 2024}

\section{堆排序}
本次作业思路较为简单,直接使用std中的make\_heap和pop\_heap即可实现,具体来说,先将数组内容传入,然后建堆,每次都用pop\_heap将最大值置于末端,然后修改堆的范围,并不断循环,最后数组就会呈现出正序,这是因为建堆默认是一个大根堆,较大数会处于尾端。\\
\indent 测试来说,我一共建立了8个数组,生成了4种不同的序列,每种序列会放在两个数组内,分别让自己设计的排序函数和std中排序函数运行,并用chrono带的时间节点记录函数运行前的时间和运行后的时间,最后计算出时间间隔并输出,最后分别输出并比较,其中左侧为自设计函数,右侧为std的堆排序。最后用check函数检验自设计的堆排序是否正确,如果正确则输出correct。

\section{测试结果}
\begin{center}
\begin{tabular}{|c|c|c|}
\hline
 & my heapsort time & std::heapsort time \\
\hline
random sequence & 560ms & 585ms \\
\hline
ordered sequence & 437ms & 566ms \\
\hline
reverse sequence & 453ms & 415ms \\
\hline
repetitive sequence & 477ms & 557ms \\
\hline
\end{tabular}
\end{center}
\indent 而且通过输出结果发现,排序全部正确。输出结果为四个"correct"(std的排序无需检测正确性)
\end{document}

%%% Local Variables: 
%%% mode: latex
%%% TeX-master: t
%%% End: 
