\documentclass[UTF8]{ctexart}
\usepackage{geometry, CJKutf8}
\geometry{margin=1.5cm, vmargin={0pt,1cm}}
\setlength{\topmargin}{-1cm}
\setlength{\paperheight}{29.7cm}
\setlength{\textheight}{25.3cm}

% useful packages.
\usepackage{xcolor}
\usepackage{amsfonts}
\usepackage{amsmath}
\usepackage{amssymb}
\usepackage{amsthm}
\usepackage{enumerate}
\usepackage{graphicx}
\usepackage{multicol}
\usepackage{fancyhdr}
\usepackage{layout}
\usepackage{listings}
\usepackage{float, caption}

\lstset{  
  basicstyle=\ttfamily,  
  backgroundcolor=\color{white},  
  frame=single,  
  framesep=5pt,  
  rulecolor=\color{black},   
  breaklines=true,  
  xleftmargin=\parindent,  
  xrightmargin=\parindent  
}  

% some common command
\newcommand{\dif}{\mathrm{d}}
\newcommand{\avg}[1]{\left\langle #1 \right\rangle}
\newcommand{\difFrac}[2]{\frac{\dif #1}{\dif #2}}
\newcommand{\pdfFrac}[2]{\frac{\partial #1}{\partial #2}}
\newcommand{\OFL}{\mathrm{OFL}}
\newcommand{\UFL}{\mathrm{UFL}}
\newcommand{\fl}{\mathrm{fl}}
\newcommand{\op}{\odot}
\newcommand{\Eabs}{E_{\mathrm{abs}}}
\newcommand{\Erel}{E_{\mathrm{rel}}}

\begin{document}

\pagestyle{fancy}
\fancyhead{}
\lhead{骆弘毅, 3210103287}
\chead{数据结构与算法第四次作业}
\rhead{Oct.21th, 2024}

\section{测试程序的设计思路}
由于在程序中调用的函数较多,此处介绍起来可能会较为冗长,因此我在程序中写了详细的注释,标明了每一步想要测试的功能。此处仅对测试程序做一个整体的介绍。我首先验证了双向链表的几个构造函数是否能够正确实现,为此我创建了一个由初始化列表赋值的链表,并通过调用List.h中的各种构造函数,验证其正确性。接着我验证了链表的size(),empty(),clear()等功能能否正确实现并在完成操作后执行打印链表的操作。其中由于打印链表的操作比较频繁,因此我单独写了打印链表的函数。值得注意的是,为了验证含右值的链表操作是否能够正确实现,我写了一个create\_Mylist()的函数,其返回值为链表的一个副本,同时也是一个临时值,因此可以验证右值相关功能。接下来我开始验证一些有关迭代器的功能,包括首尾迭代器,自增自减运算符,判断运算符,取值运算符,插入删除等等。此处有关右值的功能我采用了数值运算表达式。在所有操作完成后,我均采用打印首尾元素或者打印完整链表的操作验证其正确性。以下是源代码,在每一段测试代码后面我均写了详细注释标明此处的测试目的。

\begin{lstlisting}[language=C++, caption={List.cpp},keywordstyle=\color{cyan},commentstyle=\color{green},stringstyle=\color{red},breaklines=true]
#include "List.h"
#include <iostream>
List<int> create_Mylist(List<int>& initial_list)
{
   initial_list.push_back(11);      //调用的功能包括begin(),end(),insert()等函数
   initial_list.push_front(-1);
   return initial_list;
}
//自定义函数,向头尾部各插入一个数,并返回一个临时值

void print_List(List<int>& pl)
{
   List<int>::iterator iter=pl.begin();
   while (iter!=pl.end())
   {
      std::cout<<*iter<<' ';
      iter++;
   }
   std::cout<<std::endl;
}//打印链表


int main()
{
    List<int> List1;                   //双向链表的缺省构造函数,且该过程调用私有成员的init()
    List<int> List2={1,2,3,4,5};      //使用初始化链表的构造函数
    print_List(List2);
    List<int> List3(List2);            //测试拷贝构造函数
     print_List(List3);
\end{lstlisting}
\begin{lstlisting}[language=C++, keywordstyle=\color{cyan},commentstyle=\color{green},stringstyle=\color{red},breaklines=true]
    List1=List3;                     //测试赋值函数(copy and swap)
   print_List(List1);
    List<int> List4(create_Mylist(List2)); //测试移动构造函数,函数返回值为右值
    List<int>::const_iterator iter1; //只读迭代器的缺省构造函数
    int size4=List4.size();           //测试size()函数
    std::cout<<size4<<std::endl;
    bool emp4=List4.empty();          //测试empty()函数
    std::cout<<emp4<<std::endl;
    List4.clear();                   //测试clear()函数
    std::cout<<List4.empty()<<std::endl;

    List<int> List5={1,3,5};
    std::cout<<List5.front()<<std::endl;
    std::cout<<List5.back()<<std::endl;  //测试非常量对象的front(),back()函数
    //同时也调用了非常量对象的begin()和end()

    const List<int> List6={2,4,6};
    std::cout<<List6.front()<<std::endl;
    std::cout<<List6.back()<<std::endl;  //测试常量对象的front(),back()函数
 //同时也调用了常量对象的begin()和end()

    List5.push_front(1+2+3);
    List5.push_back(20-10+8);  //测试含右值的push_front()和push_back()
    std::cout<<List5.front()<<std::endl;
    std::cout<<List5.back()<<std::endl;

    List5.pop_front();
    List5.pop_back();
    std::cout<<List5.front()<<std::endl;
    std::cout<<List5.back()<<std::endl; //测试pop_front(),pop_back()

    List<int> List7={7,8,9};
    List7.insert(--List7.end(),1+9); //调用前置自减
    List7.insert(List7.begin(),3+3); //测试insert()的右值版本 

     print_List(List7);
 

    List7.erase(++List7.begin(),List7.end()); //测试erase()的范围擦除以及前置自增
     print_List(List7);
\end{lstlisting}
\begin{lstlisting}[language=C++, keywordstyle=\color{cyan},commentstyle=\color{green},stringstyle=\color{red},breaklines=true]
    List<int> List8={10,20,30};
    List<int>::iterator iter8(List8.begin());
    iter8++;
    std::cout<<*iter8<<std::endl;
    iter8--;
    std::cout<<*iter8<<std::endl;  //测试带参数的iterator构造函数和后置自增自减
    
    List<int>::iterator iter9(List8.begin());
    List<int>::iterator iter10(List8.end());
    iter9++;
    iter10--;
    if (iter9==iter10) std::cout<<1<<std::endl;

    if (iter9!=iter10) std::cout<<0<<std::endl;//测试两个判断符

    const List<int> List10={1,2,3,5,8};
    std::cout<<(List10.begin()!=List10.end())<<std::endl;
    std::cout<<(List10.begin()==List10.end())<<std::endl;
    std::cout<<*List10.begin()<<std::endl;//测试静态的判断符

    List<int>::const_iterator iter11(List10.begin());
    List<int>::const_iterator iter12(List10.end());
    iter12--;
    iter11++;
    iter12--;
    std::cout<<*iter11<<' '<<*iter12<<std::endl;
    --iter11;
    ++iter12;
    std::cout<<*iter11<<' '<<*iter12<<std::endl;//测试常量形式下的自增自减函数(包括前置后置)
    return 0;
}
\end{lstlisting}

\section{测试的结果}
测试结果一切正常。和预期结果完全一致。
我用 valgrind 进行测试,发现没有发生内存泄露。\\
结果显示 All heap blocks were freed -- no leaks are possible


\end{document}

%%% Local Variables: 
%%% mode: latex
%%% TeX-master: t
%%% End: 
